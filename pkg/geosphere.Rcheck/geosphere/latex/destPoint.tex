\HeaderA{destPoint}{Destination given bearing and distance, when following a great circle}{destPoint}
\keyword{spatial}{destPoint}
\begin{Description}\relax
Calculate the destination point travelling along a (shortest distance) great circle arc, given a start point, initial bearing, and distance.
\end{Description}
\begin{Usage}
\begin{verbatim}
destPoint(p, brng, d, r = 6378137)
\end{verbatim}
\end{Usage}
\begin{Arguments}
\begin{ldescription}
\item[\code{p}] longitude/latitude of point(s); can be a vector of two numbers, a matrix of 2 columns (first one is longitude, second is latitude) or a spatialPoints* object
\item[\code{brng}] bearing
\item[\code{d}] distance
\item[\code{r}] radius of the earth; default = 6378137 m
\end{ldescription}
\end{Arguments}
\begin{Value}
A pair of coordinates (longitude/latitude)
\end{Value}
\begin{Note}\relax
The bearing changes continuously when traveling along a great circle line. Therefore, thbe final bearing is not the same as the initial bearing. You can comute the final bearing with \code{finalBearing} (see examples, below)
\end{Note}
\begin{Author}\relax
Chris Veness; ported to R by Robert Hijmans
\end{Author}
\begin{References}\relax
\url{http://www.movable-type.co.uk/scripts/latlong.html}

\url{http://williams.best.vwh.net/ftp/avsig/avform.txt}
\end{References}
\begin{Examples}
\begin{ExampleCode}
p <- c(5,52)
d <- destPoint(p,30,10000)

#final bearing, when arriving at endpoint: 
finalBearing(d, p)

\end{ExampleCode}
\end{Examples}

