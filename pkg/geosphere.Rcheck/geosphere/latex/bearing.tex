\HeaderA{bearing}{Bearing}{bearing}
\keyword{spatial}{bearing}
\begin{Description}\relax
Get the initial bearing (direction to travel in) to go from point1 to point2 following the shortest path (a great circle). Note that bearings change continuously while traveling along a great circle. A route with constant bearing is a rhumb line (see \code{\LinkA{brngRhumb}{brngRhumb}}.
\end{Description}
\begin{Usage}
\begin{verbatim}
bearing(p1, p2)
\end{verbatim}
\end{Usage}
\begin{Arguments}
\begin{ldescription}
\item[\code{p1}] longitude/latitude of point(s); can be a vector of two numbers, a matrix of 2 columns (first one is longitude, second is latitude) or a spatialPoints* object
\item[\code{p2}] as above. Should have same length as p1, or a single point (or vice versa when p1 is a single point
\end{ldescription}
\end{Arguments}
\begin{Value}
A bearing in degrees
\end{Value}
\begin{Author}\relax
Chris Veness; ported to R by Robert Hijmans;
\end{Author}
\begin{References}\relax
\url{http://www.movable-type.co.uk/scripts/latlong.html}

\url{http://williams.best.vwh.net/ftp/avsig/avform.txt}
\end{References}
\begin{SeeAlso}\relax
\code{ \LinkA{brngRhumb}{brngRhumb}  }
\end{SeeAlso}
\begin{Examples}
\begin{ExampleCode}
bearing(c(0,0),c(90,90))
\end{ExampleCode}
\end{Examples}

