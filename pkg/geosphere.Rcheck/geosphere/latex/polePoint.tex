\HeaderA{polePoint}{Highest latitude on a great circle}{polePoint}
\keyword{spatial}{polePoint}
\begin{Description}\relax
Given a latitude and an initial bearing, what is the polar-most point that will be reached when following a great circle? Computed with Clairaut's formula.
\end{Description}
\begin{Usage}
\begin{verbatim}
polePoint(lat, brng)
\end{verbatim}
\end{Usage}
\begin{Arguments}
\begin{ldescription}
\item[\code{lat}] latitude of point(s)
\item[\code{brng}] bearing
\end{ldescription}
\end{Arguments}
\begin{Value}
A pair of coordinates (longitude/latitude)
\end{Value}
\begin{Author}\relax
Chris Veness; ported to R by Robert Hijmans
\end{Author}
\begin{References}\relax
\url{http://williams.best.vwh.net/ftp/avsig/avform.txt}

\url{http://www.movable-type.co.uk/scripts/latlong.html}
\end{References}
\begin{Examples}
\begin{ExampleCode}
polePoint(c(5,52),30)
\end{ExampleCode}
\end{Examples}

