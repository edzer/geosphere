\HeaderA{distVincentyEllipsoid}{'Vincenty' (ellipsoid) great circle distance}{distVincentyEllipsoid}
\keyword{spatial}{distVincentyEllipsoid}
\begin{Description}\relax
The shortest distance between two points (i.e., the 'great-circle-distance' or 'as the crow flies'), according to the 'Vincenty (ellipsoid)' method. 
This method uses an ellipsoid and the results are very accurate. The method is computationally more intensive than the other great-circled methods in this package.
\end{Description}
\begin{Usage}
\begin{verbatim}
distVincentyEllipsoid(p1, p2, a=6378137, b=6356752.3142, f=1/298.257223563)
\end{verbatim}
\end{Usage}
\begin{Arguments}
\begin{ldescription}
\item[\code{p1}] longitude/latitude of point(s) 1; can be a vector of two numbers, a matrix of 2 columns (first one is longitude, second is latitude) or a spatialPoints* object
\item[\code{p2}] as above. Should have same length as p1, or a single point (or vice versa when p1 is a single point
\item[\code{a}] Equatorial axis of ellipsoid
\item[\code{b}] Polar axis of ellipsoid
\item[\code{f}] Inverse flattening of ellipsoid
\end{ldescription}
\end{Arguments}
\begin{Details}\relax
The WGS84 ellipsoid is used by default. It is the best available global ellipsoid, but for some areas other ellipsoids could be preferable, or even necessary if you work with a printed map that refers to that ellipsoid. Here are parameters for some commonly used ellipsoids:

\Tabular{rllll}{
& \code{ ellipsoid          } & \code{ a            } & \code{ b              } & \code{ f               } \\
& \code{ WGS84                  } & \code{ 6378137      } & \code{ 6356752.3142   } & \code{ 1/298.257223563 } \\
& \code{ GRS80                  } & \code{ 6378137      } & \code{ 6356752.3141   } & \code{ 1/298.257222101 } \\
& \code{ GRS67              } & \code{ 6378160      } & \code{ 6356774.719    } & \code{ 1/298.25        } \\
& \code{ Airy 1830          } & \code{ 6377563.396  } & \code{ 6356256.909    } & \code{ 1/299.3249646   } \\
& \code{ Bessel 1841        } & \code{ 6377397.155  } & \code{ 6356078.965    } & \code{ 1/299.1528434   } \\
& \code{ Clarke 1880        } & \code{ 6378249.145  } & \code{ 6356514.86955  } & \code{ 1/293.465       } \\
& \code{ Clarke 1866        } & \code{ 6378206.4    } & \code{ 6356583.8      } & \code{ 1/294.9786982   } \\
& \code{ International 1924 } & \code{ 6378388      } & \code{ 6356911.946    } & \code{ 1/297           } \\
& \code{ Krasovsky 1940     } & \code{ 6378245      } & \code{ 6356863        } & \code{ 1/298.2997381   } \\
}

more info: \url{http://en.wikipedia.org/wiki/Reference_ellipsoid}
\end{Details}
\begin{Value}
Distance value in the same units as the ellipsoid (default is meters)
\end{Value}
\begin{Author}\relax
Chris Veness and Robert Hijmans
\end{Author}
\begin{References}\relax
Vincenty, T. 1975. Direct and inverse solutions of geodesics on the ellipsoid with application of nested equations. Survey Review Vol. 23, No. 176, pp88-93.
Available here: \url{http://www.movable-type.co.uk/scripts/latlong-vincenty.html}

\url{http://www.movable-type.co.uk/scripts/latlong-vincenty.html}

\url{http://en.wikipedia.org/wiki/Great_circle_distance}
\end{References}
\begin{SeeAlso}\relax
\code{\LinkA{distVincentySphere}{distVincentySphere}, \LinkA{distVincentySphere}{distVincentySphere}, \LinkA{distVincentySphere}{distVincentySphere}}
\end{SeeAlso}
\begin{Examples}
\begin{ExampleCode}
distVincentyEllipsoid(c(0,0),c(90,90))
# on a 'Clarke 1880' ellipsoid
distVincentyEllipsoid(c(0,0),c(90,90), a=6378249.145, b=6356514.86955, f=1/293.465)
\end{ExampleCode}
\end{Examples}

