\HeaderA{greatCircleIntermediat}{Intermediate points on a great circle}{greatCircleIntermediat}
\aliasA{greatCircleIntermediate}{greatCircleIntermediat}{greatCircleIntermediate}
\keyword{spatial}{greatCircleIntermediat}
\begin{Description}\relax
Get intermediate points on a great circle inbetween the two points used to define the circle
\end{Description}
\begin{Usage}
\begin{verbatim}
greatCircleIntermediate(p1, p2, n=50) 
\end{verbatim}
\end{Usage}
\begin{Arguments}
\begin{ldescription}
\item[\code{p1}] Longitude/latitude of a single point; can be a vector of two numbers, a matrix of 2 columns (first one is longitude, second is latitude) or a spatialPoints* object
\item[\code{p2}] As above
\item[\code{n}] The requested number of points on the Great Circle
\end{ldescription}
\end{Arguments}
\begin{Value}
a matrix of points
\end{Value}
\begin{Author}\relax
Robert Hijmans based on code by Ed Williams
\end{Author}
\begin{References}\relax
\url{http://williams.best.vwh.net/avform.htm#Intermediate}
\end{References}
\begin{Examples}
\begin{ExampleCode}
greatCircleIntermediate(c(5,52), c(-120,37), n=10)
\end{ExampleCode}
\end{Examples}

