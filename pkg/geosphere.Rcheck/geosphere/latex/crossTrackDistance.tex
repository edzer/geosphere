\HeaderA{crossTrackDistance}{Cross Track Distance}{crossTrackDistance}
\keyword{spatial}{crossTrackDistance}
\begin{Description}\relax
The cross track distance (or cross track error) is the distance of a point from a great-circle path. The great circle path is defined by \code{p1} and \code{p2}, while \code{p3} is the point away from the path.
\end{Description}
\begin{Usage}
\begin{verbatim}
crossTrackDistance(p1, p2, p3, r=6378137)
\end{verbatim}
\end{Usage}
\begin{Arguments}
\begin{ldescription}
\item[\code{p1}] Start of great circle path. Longitude/latitude of point(s); can be a vector of two numbers, a matrix of 2 columns (first one is longitude, second is latitude) or a spatialPoints* object
\item[\code{p2}] End of great circle path. As above. Should have same length as p1, or a single point (or vice versa when p1 is a single point
\item[\code{p3}] Point away from the great cricle path. As for p2
\item[\code{r}] radius of the earth; default = 6378137
\end{ldescription}
\end{Arguments}
\begin{Value}
A distance in units of \code{r} (default is meters)

The sign indicates which side of the path p3 is on. Positive  means right of the course from p1 to p2, negative means left.
\end{Value}
\begin{Author}\relax
Chris Veness and Robert Hijmans
\end{Author}
\begin{References}\relax
\url{http://www.movable-type.co.uk/scripts/latlong.html}

\url{http://williams.best.vwh.net/ftp/avsig/avform.txt}
\end{References}
\begin{SeeAlso}\relax
\code{ \LinkA{alongTrackDistance}{alongTrackDistance}  }
\end{SeeAlso}
\begin{Examples}
\begin{ExampleCode}
crossTrackDistance(c(0,0),c(90,90),c(80,80))
\end{ExampleCode}
\end{Examples}

