\HeaderA{midPoint}{Mid-point}{midPoint}
\keyword{spatial}{midPoint}
\begin{Description}\relax
Mid-point between two points along a great circle
\end{Description}
\begin{Usage}
\begin{verbatim}
midPoint(p1, p2)
\end{verbatim}
\end{Usage}
\begin{Arguments}
\begin{ldescription}
\item[\code{p1}] longitude/latitude of point(s); can be a vector of two numbers, a matrix of 2 columns (first one is longitude, second is latitude) or a spatialPoints* object
\item[\code{p2}] as above. Should be of same length of p1, or a single point (or vice versa when p1 is a single point
\end{ldescription}
\end{Arguments}
\begin{Details}\relax
Just as the initial bearing may vary from the final bearing, the midpoint may not be located half-way between latitudes/longitudes; the midpoint between 35N,45E and 35N,135E is around 45N,90E.
\end{Details}
\begin{Value}
A pair of coordinates (longitude/latitude)
\end{Value}
\begin{Author}\relax
Chris Veness; ported to R by Robert Hijmans
\end{Author}
\begin{References}\relax
\url{http://www.movable-type.co.uk/scripts/latlong.html}

\url{http://en.wikipedia.org/wiki/Great_circle_distance}
\end{References}
\begin{Examples}
\begin{ExampleCode}
midPoint(c(0,0),c(90,90))  
\end{ExampleCode}
\end{Examples}

