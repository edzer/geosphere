\HeaderA{greatCircleIntersect}{Intersections of two great circles}{greatCircleIntersect}
\keyword{spatial}{greatCircleIntersect}
\begin{Description}\relax
Get the two points where two great cricles cross each other. Great circles are defined by two points on it.
\end{Description}
\begin{Usage}
\begin{verbatim}
greatCircleIntersect(p1, p2, p3, p4) 
\end{verbatim}
\end{Usage}
\begin{Arguments}
\begin{ldescription}
\item[\code{p1}] Longitude/latitude of a single point; can be a vector of two numbers, a matrix of 2 columns (first one is longitude, second is latitude) or a spatialPoints* object
\item[\code{p2}] As above
\item[\code{p3}] As above
\item[\code{p4}] As above
\end{ldescription}
\end{Arguments}
\begin{Value}
two points for each pair of great circles
\end{Value}
\begin{Author}\relax
Robert Hijmans, based on equations by Ed Williams (see reference)
\end{Author}
\begin{References}\relax
\url{http://williams.best.vwh.net/intersect.htm}
\end{References}
\begin{Examples}
\begin{ExampleCode}
p1 <- c(5,52); p2 <- c(-120,37); p3 <- c(-60,0); p4 <- c(0,70)
greatCircleIntersect(p1,p2,p3,p4)
\end{ExampleCode}
\end{Examples}

