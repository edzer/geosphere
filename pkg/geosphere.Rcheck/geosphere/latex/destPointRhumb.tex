\HeaderA{destPointRhumb}{Destination along a rhumb line}{destPointRhumb}
\keyword{spatial}{destPointRhumb}
\begin{Description}\relax
Calculate the destination point when travelling along a 'rhumb line' (loxodrome), given a start point, bearing, and distance.
\end{Description}
\begin{Usage}
\begin{verbatim}
destPointRhumb(p, brng, dist, r = 6378137)
\end{verbatim}
\end{Usage}
\begin{Arguments}
\begin{ldescription}
\item[\code{p}] longitude/latitude of point(s); can be a vector of two numbers, a matrix of 2 columns (first one is longitude, second is latitude) or a spatialPoints* object
\item[\code{brng}] bearing in degrees
\item[\code{dist}] distance; in the same unit as \code{r} (default is meters)
\item[\code{r}] radius of the earth; default = 6378137 m
\end{ldescription}
\end{Arguments}
\begin{Value}
Coordinates (longitude/latitude) of a point
\end{Value}
\begin{Author}\relax
Chris Veness; ported to R by Robert Hijmans
\end{Author}
\begin{References}\relax
\url{http://www.movable-type.co.uk/scripts/latlong.html}
\end{References}
\begin{SeeAlso}\relax
\code{ \LinkA{destPoint}{destPoint}  }
\end{SeeAlso}
\begin{Examples}
\begin{ExampleCode}
destPointRhumb(c(0,0), 30, 100000, r = 6378137)
\end{ExampleCode}
\end{Examples}

