\HeaderA{alongTrackDistance}{Along Track Distance}{alongTrackDistance}
\keyword{spatial}{alongTrackDistance}
\begin{Description}\relax
The along track distance is the distance from the start point (p1) to the closest point on the path to a third point (p3), 
following a great circle path defined by points p1 and p2
\end{Description}
\begin{Usage}
\begin{verbatim}
alongTrackDistance(p1, p2, p3, r=6378137)
\end{verbatim}
\end{Usage}
\begin{Arguments}
\begin{ldescription}
\item[\code{p1}] longitude/latitude of point(s); can be a vector of two numbers, a matrix of 2 columns (first one is longitude, second is latitude) or a spatialPoints* object
\item[\code{p2}] as above. Should have same length as p1, or a single point (or vice versa when p1 is a single point
\item[\code{p3}] as above
\item[\code{r}] radius of the earth; default = 6378137m
\end{ldescription}
\end{Arguments}
\begin{Value}
A distance in units of r (default is meters)
\end{Value}
\begin{Author}\relax
Chris Veness and Robert Hijmans
\end{Author}
\begin{SeeAlso}\relax
\code{ \LinkA{alongTrackDistance}{alongTrackDistance}  }
\end{SeeAlso}
\begin{Examples}
\begin{ExampleCode}
alongTrackDistance(c(0,0),c(90,90),c(80,80))
\end{ExampleCode}
\end{Examples}

