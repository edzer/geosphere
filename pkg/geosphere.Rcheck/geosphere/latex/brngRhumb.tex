\HeaderA{brngRhumb}{Rhumbline bearing}{brngRhumb}
\keyword{spatial}{brngRhumb}
\begin{Description}\relax
Bearing (direction of travel) along a rhumb line. Unlike a great circle, a rhumb line is a line of constant bearing.
\end{Description}
\begin{Usage}
\begin{verbatim}
brngRhumb(p1, p2)
\end{verbatim}
\end{Usage}
\begin{Arguments}
\begin{ldescription}
\item[\code{p1}] longitude/latitude of point(s); can be a vector of two numbers, a matrix of 2 columns (first one is longitude, second is latitude) or a spatialPoints* object
\item[\code{p2}] as above. Should have same length as p1, or a single point (or vice versa when p1 is a single point
\end{ldescription}
\end{Arguments}
\begin{Value}
A bearing in degrees
\end{Value}
\begin{Author}\relax
Chris Veness; ported to R by Robert Hijmans
\end{Author}
\begin{References}\relax
\url{http://www.movable-type.co.uk/scripts/latlong.html}
\end{References}
\begin{SeeAlso}\relax
\code{ \LinkA{bearing}{bearing},  \LinkA{bearing}{bearing}  }
\end{SeeAlso}
\begin{Examples}
\begin{ExampleCode}
brngRhumb(c(0,0),c(90,90))
\end{ExampleCode}
\end{Examples}

