\HeaderA{distRhumb}{Distance along a 'rhumb line'}{distRhumb}
\keyword{spatial}{distRhumb}
\begin{Description}\relax
A 'rhumb line' (or loxodrome) is a path of constant bearing, which crosses all meridians at the same angle.
\end{Description}
\begin{Usage}
\begin{verbatim}
distRhumb(p1, p2, r=6378137)
\end{verbatim}
\end{Usage}
\begin{Arguments}
\begin{ldescription}
\item[\code{p1}] longitude/latitude of point(s) 1; can be a vector of two numbers, a matrix of 2 columns (first one is longitude, second is latitude) or a spatialPoints* object
\item[\code{p2}] as above. Should be of same length of p1, or a single point (or vice versa when p1 is a single point
\item[\code{r}] radius of the earth; default = 6378137 m
\end{ldescription}
\end{Arguments}
\begin{Details}\relax
Sailors used to (and sometimes still) navigate along rhumb lines since it is easier to follow a constant compass bearing than to continually adjust the bearing as is needed to follow a great circle, though they are normally longer than great-circle (orthodrome) routes. Rhumb lines are straight lines on a Mercator Projection map.
If you maintain a constant bearing along a rhumb line, you will gradually spiral in towards one of the poles.
\end{Details}
\begin{Value}
distance value in units of r (default=meters)
\end{Value}
\begin{Author}\relax
Chris Veness; ported to R by Robert Hijmans
\end{Author}
\begin{References}\relax
\url{http://www.movable-type.co.uk/scripts/latlong.html}
\end{References}
\begin{SeeAlso}\relax
\code{\LinkA{distCosine}{distCosine}, \LinkA{distCosine}{distCosine}, \LinkA{distCosine}{distCosine}}
\end{SeeAlso}
\begin{Examples}
\begin{ExampleCode}
distRhumb(c(0,0),c(90,90))
\end{ExampleCode}
\end{Examples}

