\HeaderA{radialIntersect}{Intersecting radials}{radialIntersect}
\keyword{spatial}{radialIntersect}
\begin{Description}\relax
Intersection between two lines defined by their point of origin and true bearing
\end{Description}
\begin{Usage}
\begin{verbatim}
radialIntersect(p1, brng1, p2, brng2) 
\end{verbatim}
\end{Usage}
\begin{Arguments}
\begin{ldescription}
\item[\code{p1}] Longitude/latitude of point(s); can be a vector of two numbers, a matrix of 2 columns (first one is longitude, second is latitude) or a spatialPoints* object
\item[\code{brng1}] True bearing from p1
\item[\code{p2}] As above. Should have same length as p1, or a single point (or vice versa when p1 is a single point
\item[\code{brng2}] True bearing from p2
\end{ldescription}
\end{Arguments}
\begin{Value}
a point
\end{Value}
\begin{Author}\relax
Robert Hijmans based on code by Ed Williams
\end{Author}
\begin{References}\relax
\url{http://williams.best.vwh.net/avform.htm#Intersection}
\end{References}
\begin{Examples}
\begin{ExampleCode}
radialIntersect(c(10,0), 10, c(-10,0), 10)
\end{ExampleCode}
\end{Examples}

