\HeaderA{distVincentySphere}{'Vincenty' (sphere) great circle distance}{distVincentySphere}
\keyword{spatial}{distVincentySphere}
\begin{Description}\relax
The shortest distance between two points (i.e., the 'great-circle-distance' or 'as the crow flies'), according to the 'Vincenty (sphere)' method. 
This method assumes a spherical earth, ignoring ellipsoidal effects and it is less accurate then the \code{distVicentyEllipsoid} method.
\end{Description}
\begin{Usage}
\begin{verbatim}
distVincentySphere(p1, p2, r=6378137)
\end{verbatim}
\end{Usage}
\begin{Arguments}
\begin{ldescription}
\item[\code{p1}] longitude/latitude of point(s) 1; can be a vector of two numbers, a matrix of 2 columns (first one is longitude, second is latitude) or a spatialPoints* object
\item[\code{p2}] as above. Should have same length as p1, or a single point (or vice versa when p1 is a single point
\item[\code{r}] radius of the earth; default = 6378137 m
\end{ldescription}
\end{Arguments}
\begin{Value}
Distance value in the same unit as \code{r} (default is meters)
\end{Value}
\begin{Author}\relax
Robert Hijmans
\end{Author}
\begin{References}\relax
\url{http://en.wikipedia.org/wiki/Great_circle_distance}
\end{References}
\begin{SeeAlso}\relax
\code{\LinkA{distVincentySphere}{distVincentySphere}, \LinkA{distVincentySphere}{distVincentySphere}, \LinkA{distVincentySphere}{distVincentySphere}}
\end{SeeAlso}
\begin{Examples}
\begin{ExampleCode}
distVincentySphere(c(0,0),c(90,90))
\end{ExampleCode}
\end{Examples}

